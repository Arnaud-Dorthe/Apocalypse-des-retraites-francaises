\documentclass[10pt]{article}

\usepackage[utf8]{inputenc}
\usepackage[french]{babel}
%\usepackage[T1]{fontenc}
\usepackage{lmodern}
\usepackage{hyperref}
\usepackage{amsthm}
\usepackage{amscd}
\usepackage{mathtools}
\usepackage{amssymb}
\mathtoolsset{showonlyrefs=true}
\usepackage{fullpage}
\usepackage{graphicx}

%Il faut qu’on écrive une page ou deux par contre, pour expliquer le projet. En gros un titre,   
\title{Ce que l'étude d'impact ne dit pas}

\author{Bruno Scherrer}
\author{Michaël Baudin}


\begin{document}

\maketitle

\section{Résumé}

Le 24 Janvier 2020, le gouvernement a rendu public une 
étude d'impact ayant pour objectif de présenter le projet 
de loi instituant le système universel de retraites. 
L'objectif du présent texte est de permettre de comprendre 
l'influence de cette réforme sur l'équilibre financier macro-économique 
du système de retraite. 
Nous montrons pourquoi les simulations montrent que l'âge de départ à la 
retraite augmente et que le niveau des pensions diminue, contrairement à ce que laisse 
penser l'étude d'impact. 
Ainsi, l'étude d'impact ne présente pas de résultat techniquement faux : 
elle se content de dissimuler l'effet de la réforme \emph{par omission}, 
laissant penser ce qu'elle ne dit, en fait, pas. 

\section{Modèle du simulateur officiel du COR}

Dans le but de pouvoir comprendre l'influence des changements indiqués par 
l'étude d'impact, nous souhaiterions pouvoir utiliser le simulateur du COR 
(\url{https://www.cor-retraites.fr/simulateur}). 
Ce simulateur tient compte de deux variables permettant de définir un scénario :
\begin{itemize}
\item le taux de hausse des salaires : +1\%, +1.3\%, +1.5\%, +1.8\%, 
\item le taux de chômage : 4.5%, 7\%, 10\%.
\end{itemize}


\end{document}

