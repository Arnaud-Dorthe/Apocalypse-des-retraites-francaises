\documentclass[10pt]{article}

\usepackage[utf8]{inputenc}
\usepackage[french]{babel}
%\usepackage[T1]{fontenc}
\usepackage{lmodern}
\usepackage{hyperref}
\usepackage{amsthm}
\usepackage{amscd}
\usepackage{mathtools}
\usepackage{amssymb}
\mathtoolsset{showonlyrefs=true}
\usepackage{fullpage}
\usepackage{graphicx}

%Il faut qu’on écrive une page ou deux par contre, pour expliquer le projet. En gros un titre,   
\title{Ce que l'étude d'impact ne dit pas}

\author{Bruno Scherrer}
\author{Michaël Baudin}


\begin{document}

\maketitle

\section{Résumé}

Le 24 Janvier 2020, le gouvernement a rendu public une 
étude d'impact ayant pour objectif de présenter le projet 
de loi instituant le système universel de retraites. 
L'objectif du présent texte est de permettre de comprendre 
l'influence de cette réforme sur l'équilibre financier macro-économique 
du système de retraite. 
Nous montrons pourquoi les simulations montrent que l'âge de départ à la 
retraite augmente et que le niveau des pensions diminue, contrairement à ce que laisse 
penser l'étude d'impact. 
Ainsi, l'étude d'impact ne présente pas de résultat techniquement faux : 
elle se content de dissimuler l'effet de la réforme \emph{par omission}, 
laissant penser ce qu'elle ne dit, en fait, pas. 

%%%%%%%%%%%%%%%%%%%%%%%%%%%%%%%%%%%%%%%%%%%%%%

\section{Modèle du simulateur officiel du COR}

Dans le but de pouvoir comprendre l'influence des changements indiqués par 
l'étude d'impact, nous souhaiterions pouvoir utiliser le simulateur du COR 
(\url{https://www.cor-retraites.fr/simulateur}). 
Comme nous allons le voir, l'exercice de reproduction des résultats 
de l'étude d'impact révèle les intentions des auteurs de l'étude d'impact. 

Ce simulateur tient compte de deux variables permettant de définir un scénario :
\begin{itemize}
\item le taux de hausse des salaires : +1\%, +1.3\%, +1.5\%, +1.8\%, 
\item le taux de chômage : 4.5%, 7\%, 10\%.
\end{itemize}

Les rapports du COR s'appuient la plupart du temps sur le taux 
de chômage de 7\% et prennent en compte les différents taux de hausse 
des salaires de +1\%, +1.3\%, +1.5\% à +1.8\%. 
Au contraire, l'étude d'impact ne présente généralement qu'une seule 
courbe, correspondant au taux de chômage de 7% avec une hausse des 
salaires de +1.3%. 
Ainsi, on ne peut pas connaître l'influence de ce paramètre sur les calculs 
de l'étude d'impact. 

Une fois le scénario choisi dans le simulateur du COR, 
l'utilisateur doit ajuster trois leviers : 
\begin{itemize}
\item l'âge de départ à la retraite, 
\item le taux de cotisation, 
\item le niveau des pensions par rapport aux salaires. 
\end{itemize}

En sortie, le simulateur du COR calcule :
\begin{itemize}
\item la situation financière du système de retraites, 
\item le niveau de vie des retraités, 
\item la durée de vie passée à la retraite. 
\end{itemize}

On peut utiliser ce simulateur de différentes manières, mais la 
logique qui a dominé dans le passé a consisté à se fixer un objectif 
de niveau de vie des retraités, puis à augmenter l'âge de départ ou 
le taux de cotisations, tout en élevant progressivement le niveau des pensions. 

Reproduire les simulations de l'étude d'impact avec le simulateur du COR 
est donc impossible à priori. 
D'une part, le simulateur ne présente pas le niveau de dépenses du système 
de retraites. 
Or l'objectif du gouvernement est d'abaisser ce niveau de dépenses 
(proche de 14\% en 2020) jusqu'au niveau moyen européen (proche de 12.5\%). 
D'autre part, le simulateur ne permet pas d'imposer l'équilibre financier du système 
de retraites. 
Or cet équilibre financier est l'objet du projet de loi organique. 

C'est pourquoi une inversion mathématique est nécessaire pour pouvoir reproduire 
les résultats de l'étude d'impact. 
C'est la raison pour laquelle nous avons développé un simulateur Open Source 
(\url{https://github.com/brunoscherrer/retraites}) fondé sur les mêmes équations mathématiques 
que le simulateur du COR, mais dont nous avons inversé les relations pour 
pouvoir imposer les paramètres pris en compte dans l'étude d'impact. 

%%%%%%%%%%%%%%%%%%%%%%%%%%%%%%%%%%%%%%%%%%%%%%

\section{Conclusion}

Les parlementaires qui auraient pris le soin de lire l'impressionnante 
étude d'impact (1029 pages !) en sont pour leur frais. 
En effet, le text ne présente pas les informations essentielles qui leur auraient étés 
utiles pour s'informer. 
Pire : les informations présentés dans le texte détournent l'attention de l'essentiel 
mettant en avant des détails techniques accessoires et affichant fallacieusement 
des informations soigneusement choisies. 

Les citoyens que nous sommes peuvent comprendre qu'une proposition de loi 
n'aye pas nécessairement dans le sens politique que nous préférons. 
En revanche, nous ne pouvons accepter que la décision politique soit 
prise sur la base d'informations trompeuses. 

\end{document}

